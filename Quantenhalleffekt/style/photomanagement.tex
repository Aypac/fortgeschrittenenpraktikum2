\usepackage{ifplatform}


% ******************************~~~~*****************************
%Photo quality management. This is an old try:
\newcommand\openfile{\newwrite\outputstream\immediate\openout\outputstream=photo.log
	\writetofile[linewidth]{\the\linewidth}
	\writetofile[textwidth]{\the\textwidth}
	\writetofile[dpi]{300}
	}
\newcommand\closefile{\immediate\closeout\outputstream}
\newcommand\writetofile[2][2]{\immediate\write\outputstream{#1 #2}}

\newcommand\photo[2][2]{\includegraphics[width=#1]{#2}\writetofile[#2]{#1}}

% ***************************************************************
% And this is working
% Reduce image size for fixed dpi
% Introduce \includegraphicsRS for this purpose

% for downsampling large images: \includegraphicsRS
% (\includegraphics "resize")
% \includegraphicsRS[#1][#2]{#3}
% #1 = normal \includegraphics[] - Parameter (like scale=1) {optional}
% #2 = approx part of the screen, eg half page width use 0.5 or 1/2 {optional}
%      reduces to #2*1000px
% #3 = relative path to image file WITH ending (e.g. .jpg) {required}
\pdfimageresolution 300

\newcommand{\includegraphicsRS}{}

\ifnum\pdfshellescape=1
% Yes, enabled
% IFP - image file path
% IFN - image filename only
% here resizing at #2*1000 px wide
\newread\myinput
% must add default argument [] so typeout works
\DeclareDocumentCommand{\includegraphicsRS}{ O{\@empty} O{1} m }{ %
%\renewcommand\includegraphicsRS[2][\@empty]{
	\ifwindows %not tested...
		%\immediate\write18{mkdir images; mkdir images/downsampled; IFP=#3 ; %
		%	PIX=#2 ; %
		%	echo "Processing $IFP" ; %
		%	IFN=$(basename $IFP) ; %
		%	echo "images/downsampled/$IFN" > tmpname ; %
		%	if [ ! -f images/downsampled/$IFN ]; then %
		%	echo "File images/downsampled/$IFN not found! Converting..." ; %
		%	%the \string> after the pixel width (e.g. 1000x) will prevent imagemagick from producing files larger than the original
		%	convert $IFP -resize $PIX"x\string>" -quality 85 images/downsampled/$IFN ; %
		%	else %
		%	echo "Found images/downsampled/$IFN - reusing." ; %
		%	fi ; %
		%} % end write18
	\else %If not using windows,
		% run downsampling bash script
		\immediate\write18{mkdir -p images; mkdir -p images/downsampled; IFP=#3 ; %
			PIX=#2 ; PIX=$( awk  'BEGIN { rounded = sprintf("\%.0f", ('$PIX')*1000); print rounded }' ) ;
			echo "$PIX 1000 *p" | dc ; %
			echo "Processing $IFP" ; %
			IFN=$(basename $IFP) ; %
			echo "images/downsampled/$IFN" > tmpname ; %
			if [ ! -f images/downsampled/$IFN ]; then %
			echo "File images/downsampled/$IFN not found! Converting..." ; %
			%the \string> after the pixel width (e.g. 1000x) will prevent imagemagick from producing files larger than the original
			convert $IFP -resize $PIX"x\string>" -quality 85 images/downsampled/$IFN ; %
	
			else %
			echo "Found images/downsampled/$IFN - reusing." ; %
			fi ; %
			%alternatively: bash includegraphicsRS.sh #3 #2
			%and have the code in the *.sh-file.
		} % end write18
	\fi %Else you are damned to use full-resolution-pictures. Sorry. Or you need someone using MacOs/*ix/Cygwin to create the downsampled files for you.
	%
	%
	% here we should have downsampled file
	% retrieve downsample filename first
	\immediate\openin\myinput=tmpname
	% The group localizes the change to \endlinechar
	\bgroup
	\endlinechar=-1
	\immediate\read\myinput to \localline
	% Since everything in the group is local, we have to explicitly make the
	% assignment global
	\global\let\myTmpFileName\localline
	\egroup
	\immediate\closein\myinput
	% Clean up after ourselves
	% \immediate\write18{rm -f -- tmpname}
	% finally, include downsampled image
	\includegraphics[#1]{\myTmpFileName}
} % end renewcommand
\else
% No, disabled
% in this case, \includegraphicsRS is just the usual \includegraphics
\renewcommand\includegraphicsRS[2][\@empty]{%
	\includegraphics[#1]{#2}
}
\fi

\begin{comment}
old stuff, not used any longer. Could be useful for future changes though.
\usepackage{fp}

\newlength{\TOarg} \newlength{\TOunit}
{\catcode`p=12 \catcode`t=12 \gdef\TOnum#1pt{#1}}
\newcommand\TOop[2]{\setlength{\TOarg}{#2}%
	\FPdiv\TOres{\expandafter\TOnum\the\TOarg}{\expandafter\TOnum\the\TOunit}%
	\FPround\TOres\TOres{#1}}
\newcommand{\TOspace}{\ }
\newcommand\TOpt[2][2]{\setlength{\TOunit}{1pt}\TOop{#1}{#2}\TOres\TOspace pt}
\newcommand\TOin[2][2]{\setlength{\TOunit}{1in}\TOop{#1}{#2}\TOres\TOspace in}
\newcommand\TOcm[2][2]{\setlength{\TOunit}{1cm}\TOop{#1}{#2}\TOres\TOspace cm}
\newcommand\TOmm[2][2]{\setlength{\TOunit}{1mm}\TOop{#1}{#2}\TOres\TOspace mm}
\newcommand\TOem[2][2]{\setlength{\TOunit}{1em}\TOop{#1}{#2}\TOres\TOspace em}
\newcommand\TOpx[2][2]{\setlength{\TOunit}{1px}\TOop{#1}{#2}\TOres\TOspace px}
\newcommand\TOptdeux[2][2]{\setlength{\TOunit}{1pt}\TOop{#1}{#2}\TOres}

convert $IFP -resize 50\% images/downsampled/$IFN ; %		textwidth="\the\textwidth" ; %
TW=${textwidth:0:(-2)} ; %
VARI="$TW"; %
convert -resize 2000x -units pixelspercentimeter $IFP -density 120 -pointsize 30 -draw "gravity south-east fill black text 0,5 '$VARI'" downsampled/$IFN ; %
\end{comment}


% END of \includegraphicsRS
% ***************************************************************
% ******************************~~~~*****************************
