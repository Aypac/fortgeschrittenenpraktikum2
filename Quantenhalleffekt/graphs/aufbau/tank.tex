\documentclass[border=12pt]{standalone}% remove the border key if you want a tight output without any border.

%Optical elements
%%use latex->dvi->ps(->pdf) chain.



\usepackage[utf8]{inputenc}

\usepackage{hyperref}

\usepackage{amsmath} % AMS Math Package
\usepackage{amsfonts}
\usepackage{amsthm} % Theorem Formatting
\usepackage{amssymb}	% Math symbols such as \mathbb
\usepackage{mathtools}
\usepackage{graphicx} % Allows for eps images
\usepackage{multicol} % Allows for multiple columns
\usepackage[]{units}

\usepackage{hyperref} %Links
\usepackage{url}

\usepackage{verbatim} %Fuer mehrzeilige kommentare \begin{comment} \end{comment}

\usepackage[hang]{caption} % Captions einrücken
\usepackage{subfigure}

%Some other shit
\usepackage{float}
%\floatstyle{boxed} %Puts a box around each figure
\restylefloat{figure}
%\usepackage{wrapfig}
\usepackage{microtype}



\usepackage{xparse} % For uses like \DeclareDocumentCommand{\foocmd}{ O{default1} O{default2} m }{#1~#2~#3}

%\usepackage{color}
%\definecolor{gray}{rgb}{.5,.5,.5}
%\definecolor{lightgray}{rgb}{.25,.25,.25}

%Graphics (PGF / TIKZ) **+**+**+**+**+**+**+**+**+**+**+**+**+**+
\usepackage{tikz}
\usetikzlibrary{patterns}
\tikzset{
	hatchhor/.style={pattern=horizontal lines,pattern color=#1},
	hatchhor/.default=black
}
\tikzset{
	hatchvert/.style={pattern=vertical lines,pattern color=#1},
	hatchvert/.default=black
}
\tikzset{
	hatchdiag/.style={pattern=north east lines,pattern color=#1},
	hatchdiag/.default=black
}
\tikzset{
	hatchdiag2/.style={pattern=north west lines,pattern color=#1},
	hatchdiag2/.default=black
}
%also possible grid, crosshatch, dots, crosshatch dots, fivepointed stars, sixpointed stars

\usepackage{pgfplots}
\usepgfplotslibrary{fillbetween}

% Style to select only points from #1 to #2 (inclusive)
\pgfplotsset{select coords between index/.style 2 args={
		x filter/.code={
			\ifnum\coordindex<#1\def\pgfmathresult{}\fi
			\ifnum\coordindex>#2\def\pgfmathresult{}\fi
		}
	}
}


\newcommand{\vasymptote}[2][]{
	\draw [color=gray,densely dashed,#1] ({rel axis cs:0,0} -| {axis cs:#2,0}) -- ({rel axis cs:0,1} -| {axis cs:#2,0});
}

\newcommand{\vertline}[2][]{
	\draw [#1] ({rel axis cs:0,0} -| {axis cs:#2,0}) -- ({rel axis cs:0,1} -| {axis cs:#2,0});
}
%END Graphics **+**+**+**+**+**+**+**+**+**+**+**+**+**+**+**+**+
%Stolen from http://www.dfcd.net/articles/latex/latex.html
% **#**#**#**#**#**#**#**#**#**#**#**#**#**#**#**#**#**#**#**#**#
\makeatletter % Need for anything that contains an @ command

\let\vaccent=\v % rename builtin command \v{} to \vaccent{}
\renewcommand{\v}[1]{\ensuremath{\mathbf{#1}}} % for vectors
\newcommand{\gv}[1]{\ensuremath{\mbox{\boldmath$ #1 $}}} 
% for vectors of Greek letters
\newcommand{\uv}[1]{\ensuremath{\mathbf{\hat{#1}}}} % for unit vector
\newcommand{\abs}[1]{\left| #1 \right|} % for absolute value
\newcommand{\avg}[1]{\left< #1 \right>} % for average
\let\underdot=\d % rename builtin command \d{} to \underdot{}
\renewcommand{\d}[2]{\frac{d #1}{d #2}} % for derivatives
\newcommand{\niced}[2]{\nicefrac{d #1}{d #2}} % for in-text-derivatives
\newcommand{\nicedd}[2]{\nicefrac{d^2 #1}{d #2^2}} % for double derivatives\newcommand{\dd}[2]{\frac{d^2 #1}{d #2^2}} % for in-text-double derivatives
\newcommand{\pd}[2]{\frac{\partial #1}{\partial #2}} 
% for partial derivatives
\newcommand{\pdd}[2]{\frac{\partial^2 #1}{\partial #2^2}} 
% for double partial derivatives
\newcommand{\pdc}[3]{\left( \frac{\partial #1}{\partial #2}
	\right)_{#3}} % for thermodynamic partial derivatives
\newcommand{\ket}[1]{\left| #1 \right>} % for Dirac bras
\newcommand{\bra}[1]{\left< #1 \right|} % for Dirac kets
\newcommand{\braket}[2]{\left< #1 \vphantom{#2} \right|
	\left. #2 \vphantom{#1} \right>} % for Dirac brackets
\newcommand{\matrixel}[3]{\left< #1 \vphantom{#2#3} \right|
	#2 \left| #3 \vphantom{#1#2} \right>} % for Dirac matrix elements
\newcommand{\grad}[1]{\gv{\nabla} #1} % for gradient
\let\divsymb=\div % rename builtin command \div to \divsymb
\renewcommand{\div}[1]{\gv{\nabla} \cdot #1} % for divergence
\newcommand{\curl}[1]{\gv{\nabla} \times #1} % for curl
\let\baraccent=\= % rename builtin command \= to \baraccent
\renewcommand{\=}[1]{\stackrel{#1}{=}} % for putting numbers above =
\newtheorem{prop}{Proposition}
\newtheorem{thm}{Theorem}[section]
\newtheorem{lem}[thm]{Lemma}
\theoremstyle{definition}
\newtheorem{dfn}{Definition}
\theoremstyle{remark}
\newtheorem*{rmk}{Remark}
% **#**#**#**#**#**#**#**#**#**#**#**#**#**#**#**#**#**#**#**#**#


%Equalssign with hat/corresponds to   \equalhat
\newcommand\equalhat{%
	\stackrel{\Lambda}{=}
}
%Equalssign with !
\newcommand\shallbe{%
	\stackrel{!}{=}
}
% := and =:
\newcommand{\defeq}{\vcentcolon=}
\newcommand{\eqdef}{=\vcentcolon}



%Encirecled Numbers, used in Graphics
\let\depth\relax
\def\X#1{%
	%#1%
	%\textcircled{#1}%
	\raisebox{0.9pt}{\textcircled{\raisebox{-.9pt}{#1}}}%
	%\ding{\numexpr171+#1\relax}%
}

% Style to select only points from #1 to #2 (inclusive)
\pgfplotsset{select coords between index/.style 2 args={
		x filter/.code={
			\ifnum\coordindex<#1\def\pgfmathresult{}\fi
			\ifnum\coordindex>#2\def\pgfmathresult{}\fi
		}
	}}
	
	

%\usepackage{tikz}
\usepackage{pst-optexp}
\usepackage[utf8]{inputenc}
\usepackage{tikz}
\usetikzlibrary{patterns}
\usetikzlibrary{automata,positioning}


\def\fourFwidth{2}

\begin{document}
	
	%\psset{xunit=4cm, yunit=1.6cm}
	%\addtopsstyle{Beam}{beamwidth=0.3, fillstyle=solid, fillcolor=green, opacity=0.25, arrowscale=1.8} %,linestyle=none, fillcolor=green!18!white
	
	\begin{tikzpicture}[ %
		>=stealth,
		node distance=3cm,
		on grid,
		auto
	]
	%\draw[step=1,color=lightgray] (0,0) grid (3,10);
	
	%Vakuum
	\fill[lightgray, pattern=north east lines, pattern color=red!40] (0,1) rectangle (.5,7);
	\fill[white!0, pattern=north east lines, pattern color=red!40] (2.5,1) rectangle (3,7);
	
	\begin{scope}
		\clip (3,1) arc (0:-180:1.5) (0.018,-.15) (0,1);
		\fill[white!0,pattern=north east lines, pattern color=red!40] (0,-0.5) rectangle (3,1);
	\end{scope}
	\draw (1.5,.5) node[anchor=center] {Vakuum};
	
	%Flüssiges Helium
	\fill[white!0,pattern=north west lines, pattern color=blue!60] (.5,1) rectangle (2.5,3);
	\draw (1.5,1.5) node[anchor=center,text width=1.5cm, text centered] {flüssiges Helium};
	
	%Heliumtank
	\draw[thick] (.5,7) -- (.5,1) -- (2.5,1) -- (2.5,7);
	
	%Verdampfungskammer
	\fill[blue!5] (1,7) -- (1,2) -- (2,2) -- (2,7);
	\fill[pattern=north west lines, pattern color=blue!25] (1,7) -- (1,2) -- (2,2) -- (2,7);
	\draw[thick] (1,7) -- (1,2) -- (2,2) -- (2,7);
	\draw[darkgray, thin, <-] (2,5) -- (3.4,5.5) node[anchor=west, text width=1.5cm] {Unter-druck-kammer};
	
	%Ventil
	\fill[red] (1.9,2.2) rectangle (1.8,1.9);
	\draw[darkgray, thin, <-] (1.91,1.95) -- (3.4,1.4) node[anchor=west, text width=1.5cm] {Nadel-Ventil};
	\draw[->, thin, blue!45] (1.8, 2.25) -- (1.5, 2.5);
	\draw[->, thin, blue!45] (1.85, 2.25) -- (1.6, 2.65);
	\draw[->, thin, blue!45] (1.9, 2.25) -- (1.8, 2.7);
	\draw[darkgray, thin, <-] (1.87,2.5) -- (3.4,2.6) node[anchor=west, text width=1.5cm] {vergasendes Helium};
	
	%Probenkammer
	\draw[thin] (1.3,7+.7) -- (1.3,2.8) -- (1.7,2.8) -- (1.7,7+.7) -- (1.3,7+.7);
	\draw[darkgray, thin, <-] (1.8,7+.4) -- (2.5,7+.4) node[anchor=west] {Einschub};
	
	%Probenstab
	\fill[black] (1.4,7+1.5) -- (1.4,2.9) -- (1.6,2.9) -- (1.6,7+1.5);
	\draw[darkgray, thin, <-] (1.7,7+1.15) -- (2.5,7+1.15) node[anchor=west] {Probenstab};
	%Probe
	\fill[red] (1.45,3)rectangle (1.55,3.3);
	\draw[darkgray, thin, <-] (1.625,3.15) -- (3.4,3.8) node[anchor=west, text width=1.5cm] {Probe \& Sensoren};
	
	%äußerster Tank
	\draw[thick] (3,7) -- (3,1) arc (0:-180:1.5) (0.018,-.15) (0,1) -- (0,7);
	\fill[] (3,7+.05) rectangle (0,7-.05);
	\draw[thick] (3,7+.05) rectangle (0,7-.05);
	
	%Heliumnachfüllung
	\fill[white] (2.35-1.5,7+.1) rectangle (2.15-1.5,7-.1);
	\draw[thin] (2.35-1.5,7+.1) -- (2.35-1.5,7-.1);
	\draw[thin] (2.15-1.5,7+.1) -- (2.15-1.5,7-.1);
	\draw[thin, <-] (2.25-1.5,7-.3) -- (2.25-1.5,7+.3) node[anchor=west,rotate=90] {Heliumeinlass};
	
	%zur Pumpe
	\fill[white] (1.75-.65,7+.1) rectangle (1.9-.65,7-.1);
	\draw[thin] (1.75-.65,7+.1) -- (1.75-.65,7-.1);
	\draw[thin] (1.9-.65,7+.1) -- (1.9-.65,7-.1);
	\draw[<-, thin, blue!40] (1.2, 7-.35) -- (1.25, 7-.7);
	\draw[<-, thin, blue!40] (1.15, 7-.35) -- (1.15, 7-.7);
	\draw[<-, thin, blue!40] (1.1, 7-.35) -- (1.05, 7-.7);
	\draw[thin, ->] (1.825-.65,7-.3) -- (1.825-.65,7+.3)  node[anchor=west, rotate=90] {Pumpe};
	

	
	\end{tikzpicture}
\end{document}