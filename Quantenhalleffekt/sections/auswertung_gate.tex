% !TeX root = ../praktikum.tex
% !TeX encoding = UTF-8
% !Tex spellcheck = de_DE


Um anhand der Messdaten zur Gatespannung den Abstand des 2DEG zur Probenoberfläche zu bestimmen, wurde zunächst angenommen, dass das 2DEG und das verwendete Titan-Gate einen Plattenkondensator bilden. Die beiden Schichten von GaAs und AlGaAs dazwischen bilden hierfür ein nichtleitendes Dielektrikum. 
Durch das Anlegen der Gatespannung wurde die Ladungsträgerdichte im 2DEG verändert. 
In einem Plattenkondensator der Fläche A ist die Anzahl der Ladungsträger $N_s=n_s \cdot A$ gegeben durch die Gleichung \ref{eq:kondensator_ladung}. 
Damit ergibt sich der Abstand d der Kondensatorplatten und somit der Abstand des 2DEG zur Probenoberfläche aus der Gleichung \ref{eq:kondensator_abstand}. 

Dabei kann $\epsilon \approx 12$ angenommen werden, da die verwendete Probe im wesentlichen AlGaAs mit einem Aluminiumanteil von 33 \% enthält.
Anhand der beiden Gleichungen %TODO: ref! (die beiden Gleichungen drüber verwendet!)
und der Einsatzspannung $U_{th}$ kann nun die Abhängigkeit der Ladungsträgerdichte von der Gatespannung angegeben werden durch Gleichung \ref{eq:kondens_lad_und_abst}:

\begin{equation}
n_s=\frac{\epsilon \epsilon_0}{d}(U_{Gate}-U_{th})
\end{equation}
Dieser Zusammenhang wurde in der Abbildung %TODO: GRAPHIK!
dargestellt, indem die bestimmten Ladungsträgerdichten gegen die angelegten Gatespannungen aufgetragen wurden. 
Die Ladungsträgerdichten wurden dabei analog zu den obigen Versuchsteilen aus der Näherung über die Shubnikov-de Haas-Oszilation %TODO: Oder Hall-Spannung, was dir lieber ist
bestimmt und es ergaben sich folgende Werte: %TODO: Werte angeben
Der Abstand d des 2DEG zur Probenoberfläche wurde mittels linearer Regression bestimmt. Dafür ergab sich ein Wert von $d=...$ %TODO: WERT ANGEBEN



