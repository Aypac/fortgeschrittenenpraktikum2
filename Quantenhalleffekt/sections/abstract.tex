% !TeX root = ../praktikum.tex
% !TeX encoding = UTF-8
% !Tex spellcheck = de_DE


Der Quanten-Hall-Effekt wurde 1980 von K. von Klitzing, G. Dorda sowie M.Pepper entdeckt. Er unterscheidet sich vom klassischen Hall-Effekt, dadurch, dass bei ersterem der Wert des Hallwiderstands $R_H$ nicht linear mit steigendem Magnetfeld zunimmt, sondern Plateaus ausbildet. Dabei sind die Spannungswerte dieser Plateaus unabhängig von der Probe und nehmen nur Bruchteile einer rein durch Naturkonstanten bestimmten Größe an. Dadurch wird der Effekt als Norm für den elektrischen Widerstand genutzt. Im Jahr 1985 erhielt Klitzing für die Untersuchung des Quanten-Hall-Effekts den Nobelpreis.

Im vorliegenden Versuchsprotokoll werden die theoretischen Grundlagen, die Durchführung sowie die Befunde und Schlussfolgerungen des Experiments zum Quanten-Hall-Effekt dargestellt. Dafür wird ein zweidimensionales Elektronengas auf wenige Kelvin herunter gekühlt, von einem Strom durchflossen und einem starken Magnetfeld ausgesetzt.