% !TeX root = ../praktikum.tex
% !TeX encoding = UTF-8
% !Tex spellcheck = de_DE


Der Quanten-Hall-Effekt wurde 1980 von K. von Klitzing, G. Dorda sowie M.Pepper entdeckt. Er unterscheidet sich vom klassischen Hall-Effekt, da hier der Wert des Hallwiderstands $R_H$ nicht linear mit steigendem Magnetfeld zunimmt, sondern Plateaus der Werte 
\begin{equation}
R_H=\frac{1}{\nu}\cdot 25812,8\Omega =\frac{1}{\nu} \frac{h}{e^2}
\end{equation}

ausbildet und sich dabei unabhängig von der Probe verhält. Dabei ist $\nu$ ein Füllfaktor, $h$ das Planksche Wirkungsquantum und e die Elementarladung. 
Voraussetzung für den Quanten-Hall-Effekt sind ein starkes Magnetfeld, sehr tiefe Temperaturen, sowie ein näherungsweise zweidimensionales Elektronengas im Inneren der Probe, in welchem der Effekt beobachtet wird.  
Im Jahr 1985 erhielt Klitzing für die Untersuchung des Quanten-Hall-Effekts den Nobelpreis. 

Im vorliegenden Versuchsprotokoll werden die theoretischen Grundlagen, die Durchführung sowie die Befunde und Schlussfolgerungen des Experiments zum Quanten-Hall-Effekt dargestellt.
Im Experiment wurde der elektronische Ladungstransport in einem 2DEG im Rahmen des Quanten-Hall-Effekts untersucht, wobei die Ladungsträgerdichte und -beweglichkeit in Abhängigkeit veränderlicher Variablen im Versuchsaufbau bestimmt wurden. Zudem wurde der Einfluss der Magnetfeldorientierung analysiert.

