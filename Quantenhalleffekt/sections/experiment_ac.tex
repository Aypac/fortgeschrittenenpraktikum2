% !TeX root = ../praktikum.tex
% !TeX encoding = UTF-8
% !Tex spellcheck = de_DE

Im folgendem Abschnitt wurden die Messungen des vorigen Versuchsteils im Wesentlichen wiederholt, mit dem Unterschied, dass anstatt eines Gleichstroms ein Wechselstrom die Probe durchfließt. 
%mit $U_{RMS}=\unit[1]{V}$ angelegt wurde und ein \unit[9,95]{$M\Omega$} Widerstand in Reihe geschaltet wurde.  Um einen Wechselstrom erzeugen zu können, wurden sogenannte Lock-in Verstärker genutzt. Ein Lock-in Verstärker enthält einen Funktionsgenerator, welcher anhand einer einstellbaren Frequenz eine sinusförmige Wechselspannung mit $U_{RMS}=\unit[1]{V}$ ausgibt. 
Dieser wurde mit einem Funktionsgenerator mit der Amplitude $U_{RMS}=\unit[1]{V}$ erzeugt. Um den Strom unabhängig von dem Widerstand des Hall-Streifens zu halten, musste ein hinreichend großer Widerstand in Reihe geschaltet werden. Da der Eigenwiderstand des Hall-Streifens bei einigen Kiloohm lag, wurde ein $\unit[1]{M\Omega}\pm 5\%$ Widerstand verwendet. So kann in guter Näherung angenommen werden, dass der Strom allein durch den zusätzlich angelegten Vorwiderstand zu $I=\nicefrac{U}{R}=\nicefrac{\unit[1]{V}}{\unit[1]{M\Omega}}=\unit[1]{\mu A}$ bestimmt wurde. Um die kleinen, dennoch auftretenden Schwankungen auszugleichen, wurde der Strom ebenfalls für jeden Datenpunkt aufgezeichnet.Für die Rechnungen wurde der Mittelwert dieser Werte zu $I=\unit[0,978]{\mu A}$ berechnet. Die Berechnung des Widerstandes mit den Formeln~\eqref{eq:rho_xx} und \eqref{eq:rho_xy} wurde entsprechend zeilenweise mit dem gemessenen Strom durchgeführt.

Die verwendeten Messgeräte für die Längs- und Querspannungen wurden mit dem Frequenzgenerator als Referenz verbunden. %TODO: Erklärung
\\

Die aus den Messwerten gewonnen Widerstandswerte sind für die gesamte Magnetfeldreichweite bei maximaler Magnetfeldrampe in Abbildung~\ref{fig:full_range_ac} und für den detaillierten Ausschnitt in Abbildung~\ref{fig:2T_range_ac} aufgetragen.

\begin{figure}[h]
	\centering
	\includegraphics{graphs/ac/full_range.pdf}
	\caption[Wechselstrommessung im maximalen Magnetfeldbereich]{
		Plot des aus den gemessenen Längs- und Querspannungen berechneten Widerständen eines mit Wechselstrom durchflossenen 2DES im maximalen Magnetfeldbereich und maximaler Magnetfeldrampe. Die Hall-Spannung und somit der berechnete Hall-Widerstand nimmt bei negativen Magnetfeldern negative Werte an, Aus Platzgründen wurde diese jedoch in den positiven Bereich geklappt.
	}
	\label{fig:full_range_ac}
\end{figure}

Auch bei diesen Messungen ist die Asymmetrie des Längswiderstandes bei positiven und negativen Magnetfeldes zu beobachten. Die Messungen des gesamten und des Teilbereiches sind sehr Kongruent, die des gesamten Bereiches kann jedoch die Oszillationen des Längswiderstandes im Bereich zwischen $-1$ und \unit[1]{T} nicht mehr gut auflösen. Durch die reduzierte Magnetfeldrampe der Messung im Teilbereich zwischen $-2$ und \unit[+2]{T} hat diese eine für diesen Bereich wesentlich besser geeignete Auflösung.


\begin{figure}[h]
	\centering
	\includegraphics{graphs/ac/pm2T_range.pdf}
	\caption[Höher aufgelöste Gleichstrommessung in Magnetfeldteilbereich]{
		Plot des aus den gemessenen Längs- und Querspannungen berechneten Widerständen eines mit Wechselstrom durchflossenen 2DES im reduzierten Magnetfeldbereich und geringerer Magnetfeldrampe. Die Hall-Spannung und somit der berechnete Hall-Widerstand nimmt bei negativen Magnetfeldern negative Werte an, Aus Platzgründen wurde diese jedoch in den positiven Bereich geklappt. In grau sind die entsprechenden Ergebnisse der Messung des gesamten Bereiches aus der vorherigen Abbildung unterlegt.
	}
	\label{fig:2T_range_ac}
\end{figure}

