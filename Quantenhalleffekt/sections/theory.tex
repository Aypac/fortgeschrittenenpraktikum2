% !TeX root = ../praktikum.tex
% !TeX encoding = UTF-8
% !Tex spellcheck = de_DE

\begin{align}
	\vec{j} = \sigma \cdot \vec{E} & & \Leftrightarrow & & \vec{E} = \rho \cdot \vec{j}
	\label{eq:u2rho}
\end{align}

\begin{equation}
	n_s=\frac{I}{e} \cdot \left( \d{U_{Hall}}{B} \right)^{-1}
	\label{eq:ladungsdichte_steig}
\end{equation}

\begin{equation}
\mu=\frac{1}{n_se}\frac{I}{U_{xx}}\frac{L}{W}
\label{eq:bewegl_masse}
\end{equation}


Plattenkondensator der Fläche A: $N_s=n_s \cdot A$ 
\begin{equation}
N_s=C \cdot U_{Gate}
\label{eq:kondensator_ladung}
\end{equation}

\begin{equation}
C=\frac{\epsilon \epsilon_0 A}{d}
\label{eq:kondensator_abstand}
\end{equation}


Aus \ref{eq:kondensator_ladung} und \ref{eq:kondensator_abstand} ergibt sich:
\begin{equation}
n_s=\frac{\epsilon \epsilon_0}{d}(U_{Gate}-U_{th})
\label{eq:kondens_lad_und_abst}
\end{equation}

mit der Einsatzspannung $U_{th}$

%TODO: Diesen Schritt von Formel 5 und 6 auf 7 hab ich im GAtesannungsteil jetzt durch refs abgekürtzt. muss ja nich alles 2mal drin stehen, dafür ist der theorieteil ja da, gell?



