% !TeX root = ../praktikum.tex
% !TeX encoding = UTF-8
% !Tex spellcheck = de_DE

In diesem Versuchsteil wurde die Abhängigkeit des Quanten-Hall-Effekts, sowie des Shubnikov-de Haas-Effekts vom Einfallswinkel des Magnetfeldes auf die Probe untersucht. Hierzu wurde ausgenutzt, dass sich der Probenstab im Einschub drehen ließ, ohne sich dabei aus der Fixierung zu lösen. 
Mit Hilfe einer Winkelscheibe auf der Abdeckung des Kryostaten und einer Markierung am Probenstab, wurde nun eine Messreihe bestehend aus mehreren Messungen für verschiedene Winkeleinstellungen der Probe zum Magnetfeld durchgeführt. Für den Winkel wurden hierbei in $10^\circ$-Schritten Werte zwischen $260^\circ$ und $10^\circ$  gewählt, sodass insgesamt eine Drehung von $90^\circ$ ausgeführt wurde. Dabei lag eine Temperatur von \unit[2]{K} an der Probe an und für jede Messung wurde das Magnetfeld analog zu den ersten Messungen von $+7,7$ bis \unit[-7,7]{T} gefahren, mit der maximalen Geschwindigkeit von \unitfrac[1]{T}{min}. Aufgrund der deutlicheren Hall-Plateaus in den Graphen wurde für diese Messreihe weiterhin die Wechselstromquelle des vorangegangenen Versuchsteils genutzt. 

Anhand der Messungen war zu erkennen, dass beide betrachteten Effekte immer weniger ausgeprägt waren und schließlich verschwanden, je weiter die Probe aus der Nullposition ausgelenkt wurde. 