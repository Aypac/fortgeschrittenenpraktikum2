% !TeX root = ../praktikum.tex
% !TeX encoding = UTF-8
% !Tex spellcheck = de_DE


Im Folgenden sind die Auswertungen der in den jeweiligen Versuchsteilen aufgenommenen Messdaten zu finden. 

\subsection{Gleichstrommessung}
\label{Auswertung_Gleichstrom}

Anhand der Messdaten aus der Gleichstrommessung, abgebildet in Graphik \ref{fig:full_range_dc} und \ref{fig:2T_range_dc}, wurde die Dichte der Ladungsträger im 2DEG, sowie deren Beweglichkeit bestimmt.

\subsubsection{Näherung über die Hall-Spannung}
\label{Über die Hall-Spannung}

 Zunächst wurde die Steigung der Hall-Spannung mittels linearer Regression aus dem klassischen Bereich der Messdaten zwischen \unit[-2]{T} und \unit[2]{T} anhand der Formel $b=\frac{dU_{Hall}}{dB}$  
bestimmt: \\

$\frac{dU_{Hall}}{dB}=  \frac{m^2}{s}$  %TODO: ZAHLENWERTE! 
\\
Hieraus ließ sich mit der Formel 
\\
$n_s=\frac{I}{e}\frac{1}{\frac{dU_{Hall}}{dB}}$ 
\\
und dem bekannten Strom von $I=\unit[100]{nA}$ die Ladungsträgerdichte des 2DEG berechnen: 
\\
$n_s= \frac{1}{cm^2}$    %TODO: ZAHLENWERTE!
\\
Mit den Bekannten Maßen der Probe wurde anschließend die Beweglichkeit der Ladungsträger bestimmt. Dies erfolgte anhand der Formel \\
$\mu=\frac{1}{n_se}\frac{I}{U_{xx}}\frac{L}{W}$
\\
und mit den Probenmaßen von $600\mu m$ für die Länge L und $100\mu m$  für die Breite B. \\

$\mu= \frac{cm^2}{Vs}$  %TODO: ZAHLENWERTE!

\subsubsection{Näherung über die Shubnikov-de Haas-Oszillation}
\label{Über die SDH-Oszillation}

Eine Alternative Möglichkeit, die Ladungsträgerdichte zu berechnen, erfolgt über die Shubnikov-de Haas-Oszillation. Hierzu wurde die Längsspannung über der Probe 
%$U_{Längs}$
 gegen $\frac{1}{B}$ aufgetragen und jedem Minimum der Oszillation ein Füllfaktor $\nu$ zugeordnet. Dies ist in Graphik %TODO: GRAPHIK 
zu sehen. 

Mit der Steigung der Geraden von $b=...$  %TODO: ZAHLENWERTE!
ließ sich analog zur Näherung über die Hallspannung mit Gleichung %TODO: REF! 
die Ladungsträgerdichte bestimmen:
\\
$n_s= \frac{1}{cm^2}$  %TODO: ZAHLENWERTE!
\\
So ergibt sich aus Gleichung %TODO: REF! 
für die Beweglichkeit:
\\
$\mu= \frac{cm^2}{Vs}$  %TODO: ZAHLENWERTE! 


\subsection{Wechselstrommessung}
\label{Auswertung_Wechselstrom}

Analog zur Auswertung der Gleichstrommessung wurde auch für die aufgenommenen Messdaten der Wechselstrommessung, abgebildet in Graphik \ref{fig:full_range_ac} und \ref{fig:2T_range_ac}, die Dichte der Ladungsträger im 2DEG, sowie deren Beweglichkeit bestimmt.

\subsubsection{Näherung über die Hall-Spannung}
\label{über Hall-Spannung}

Wie in der Auswertung der Gleichstrommessung wurde auch hier die Ladungsträgerdichte und deren Beweglichkeit im 2DEG zunächst über die Steigung des klassischen Teils der Hallspannung zwischen \unit[-2]{T} und \unit[2]{T} anhand der Formel $b=\frac{dU_{Hall}}{dB}$  
bestimmt. Dabei ergab sich aufgrund des großen Vorwiderstands ein kleinerer Spannungsabfall relativ zur Spannungsquelle und somit über 
\\
$\frac{dU_{Hall}}{dB}=  \frac{m^2}{s}$  %TODO: ZAHLENWERTE! 
\\
ein geringerer Strom bei der Wechselstrommessung von $I=\unit[]{nA}$. %TODO: ZAHLENWERTE! 

Analog zur Auswertung der Gleichstrommessdaten ergaben sich hier für die gesuchten Größen folgende Werte: \\

$n_s= \frac{1}{cm^2}$  %TODO: ZAHLENWERTE!


$\mu= \frac{cm^2}{Vs}$  %TODO: ZAHLENWERTE! 


\subsubsection{Näherung über die Shubnikov-de Haas-Oszillation}
\label{über SDH-Oszillation}

Ebenfalls analog zur Gleichstrommessung wurden die gesuchten physikalischen Größen alternativ über die Shubnikov-de Haas-Oszillation berechnet. Hierzu wurde wieder die Längsspannung über der Probe 
%$U_{Längs}$
gegen $\frac{1}{B}$ aufgetragen und jedem Minimum der Oszillation ein Füllfaktor $\nu$ zugeordnet. Dies ist in Graphik %TODO: GRAPHIK 
zu sehen. 
Es ergibt sich eine Steigung der Geraden von $b=...$  %TODO: ZAHLENWERTE!
und daraus entsprechend mit Gleichung %TODO: REF
die Ladungsträgerdichte $n_s= \frac{1}{cm^2}$  %TODO: ZAHLENWERTE!
und mit Gleichung %TODO: REF
die Beweglichkeit          
$\mu= \frac{cm^2}{Vs}$ . %TODO: ZAHLENWERTE! 



\subsection{Vergleich der Messergebnisse aus Gleich- und Wechselstrommessung}
\label{Vergleich der Messergebnisse}


%TODO: - Ergebnisse auch wie Bernd in ner Tabelle gegenüberstellen und vgl.? - Hier noch ne Fehlerdiskussion, je nach Ergebnissen? (wegen unserer Asymmetrie o.ä.?)


\subsection{Winkelabhängigkeit}
\label{Winkelabhängigkeit}


\subsection{Temperaturabhängigkeit}
\label{Temperaturabhängigkeit}


\subsection{Gatespannungsabhängikeit}
\label{Gatespannungsabhängikeit}

