% !TeX root = ../praktikum.tex
% !TeX encoding = UTF-8
% !Tex spellcheck = de_DE

Ein Vergleich der Auswertungen für Gleich- und Wechselstrommessungen zeigt, dass sich die Ergebnisse jeweils um mindestens Faktor 2 unterscheiden. Dabei handelt es sich offensichtlich um einen systematischen Fehler, welcher nicht vollständig erklärt werden kann. Die Grundlegende Ursache wird jedoch auf eine Änderung der Ladungsträgerdichte in der Probe, welche zwischen den beiden Messreihen, zurückzuführen sein. Dies ließ sich bereits anhand der Messdaten selbst erkennen, da sich die Kurven der Graphen für die beiden verschiedenen Messreihen bereits um den selben Faktor in ihrer Amplitude unterscheiden.

Es wurden verschiedene Theorien aufgestellt, um diese Abweichung zu erklären. Eine elektrostatische Ladung des Gates auf etwa \unit[100]{mV} (vgl. Kapitel~\ref{ch:ausw_gate}) würde ausreichen, um diesen Effekt zu erklären. Eine Spannung in diesem Bereich kann sehr leicht durch Berührung von statisch aufgeladenen Gegenständen (z.B. Bekleidung) oder selbst von der Haut hervorgerufen werden.

Es lässt sich jedoch festhalten, dass beide Messungen in sich konsistent gewesen zu sein scheinen, was sich an der in beiden Fällen recht sauberen linearen Regression zeigt. Die Näherung über die Hall-Spannung liefert in beiden Fällen einen kleineren Fehler für die zu berechnenden physikalischen Größen. 



