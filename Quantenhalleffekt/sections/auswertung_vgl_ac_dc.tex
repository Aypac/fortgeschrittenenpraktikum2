% !TeX root = ../praktikum.tex
% !TeX encoding = UTF-8
% !Tex spellcheck = de_DE

%TODO: - Ergebnisse auch wie Bernd in ner Tabelle gegenüberstellen und vgl.? - Hier noch ne Fehlerdiskussion, je nach Ergebnissen? (wegen unserer Asymmetrie o.ä.?)


Ein Vergleich der Auswertungen für Gleich- und Wechselstrommessungen zeigt, dass sich die Ergebnisse jeweils um mindestens Faktor 2 unterscheiden. Dabei handelt es sich offensichtlich um einen systematischen Fehler, welcher nicht vollständig erklärt werden kann, jedoch auf eine Änderung der Ladungsträgerdichte in der Probe, welche zwischen den beiden Messreihen stattgefunden hat, zurückzuführen ist. Dies ließ sich bereits anhand der Messdaten selbst erkennen, da sich die Kurven der Graphen für die beiden verschiedenen Messreihen bereits um den selben Faktor in ihrer Amplitude unterscheiden. 
Aus diesem Grund lassen sich die Gleich- und Wechselstrommessungen hier leider kaum vergleichen. Was eine solche Veränderung in der Ladungsträgerdichte verursacht haben könnte, zum Beispiel eine Veränderung der Temperatur an der Probe, ist nicht festzustellen.

Grob lässt sich im Vergleich feststellen, dass die berechneten Werte für die Ladungsträgerdichte, sowie für deren Beweglichkeit jeweils in etwa in der gleichen Größenordnung liegen. Für der Näherung über die SDH-Oszillation ergaben beide Messreihen recht zufriedenstellende Ergebnisse, was sich an der in beiden Fällen recht sauberen linearen Regression zeigt. Die Näherung über die Hall-Spannung liefert in beiden Fällen einen kleineren Fehler für die zu berechnenden physikalischen Größen. 



