% !TeX root = ../praktikum.tex
% !TeX encoding = UTF-8
% !Tex spellcheck = de_DE

Im Folgenden sollte nun die Temperaturabhängigkeit des Quanten-Hall- und des Shubnikov-de Haas-Effekts überprüft werden. Ziel dabei war es, die Grenztemperatur zu finden, bis zu welcher die beiden Effekte noch zu beobachten sind. 
Dazu wurden Messungen für unterschiedliche Temperaturwerte an der Probe aufgenommen. Begonnen wurde dabei bei der Ausgangstemperatur von \unit[2]{K} und es wurden in unregelmäßigen Intervallen Temperaturen bis hin zu \unit[40]{K} gewählt. Für jede Messung wurde analog zu den vorangegangenen Versuchsteilen das Magnetfeld mit einer Geschwindigkeit von \unitfrac[1]{T}{min} hochgefahren, diesmal von jeweils 0 bis \unit[7,7]{T}. Auch hier wurde aufgrund der deutlicheren Hall-Plateaus in den Graphen weiterhin die Wechselstromquelle genutzt. 

Wie aus den Graphen %TODO: Graphen
ersichtlich, liegt die gesuchte Grenztemperatur, bis zu welcher der Quanten-Hall- und der Shubnikov-de Haas-Effekt noch zu beobachten sind, bei etwa \unit[15]{K}. %TODO: richtiger Wert?





