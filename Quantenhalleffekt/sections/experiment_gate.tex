% !TeX root = ../praktikum.tex
% !TeX encoding = UTF-8
% !Tex spellcheck = de_DE

Im letzten Versuchsabschnitt wurde der Einfluss einer Gatespannung auf die beiden zu untersuchenden Effekte beleuchtet, mit dem Ziel, aus den dadurch erfassten Daten, den Abstand des 2DEG in der Probe zur Gateelektrode zu bestimmen.
Das Anlegen eines elektrostatischen Potentials den Bandverlauf in der Probe verändern.
Für die folgenden Messungen wurden in Schritten von \unit[50]{mV} Gatespannungen von -200 bis \unit[200]{mV} gewählt.
Dabei wurde eine Temperatur von \unit[2]{K} an der Probe eingestellt und analog zu den obigen Versuchsteilen aufgrund der deutlicheren Hall-Plateaus in den Graphen die Wechselstromquelle genutzt. 


\begin{figure}[h]
	\centering
	\includegraphics[scale=1]{graphs/gate/full_range.pdf}
	\caption[Auswertung der Gatespannungsvariation]{
		Berechnete Elektronendichte $n_s$ und -beweglichkeit $\mu$ in Abhängigkeit zu der Gatespannung.
	}
	\label{fig:gate_mess}
\end{figure}

