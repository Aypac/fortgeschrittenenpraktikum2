% !TeX root = ../praktikum.tex
% !TeX encoding = UTF-8
% !Tex spellcheck = de_DE


Wie zu den Messdaten der Gleich- und Wechselstrommessungen wurden die Ladungsträgerdichten und -beweglichkeiten hier für alle verschiedenen aufgenommenen Temperaturwerte von \unit[2]{K} bis \unit[40]{K} bestimmt. Hierzu wurde die Näherung über die Shubnikov-de Haas-Oszillation gewählt. %TODO: Oder ist dir die über die HAllspannung lieber? 
In der Abbildung %TODO: GRAPHIK
sind die Shubnikov-de Haas-Oszillationen zu verschiedenen Temperaturen abgebildet. Hierbei ist deutlich zu erkennen, wie die Oszillation mit steigender Temperatur abschwächt. %TODO: bzw. die Plateaus nicht mehr zu erkennen sind, da sie in eine Gerade konstanter Steigung übergehen...? 
Die Temperatur, ab welcher der Effekt nicht mehr zu erkennen ist, liegt offenbar bei ca. \unit[15]{K}.%TODO: richig erinnert?
Ladungsträgerdichte und -beweglichkeit wurden wie in den obigen Versuchsteilen berechnet und die Ergebnisse zu den verschiedenen Temperaturen in folgender Tabelle dargestellt. %TODO: TABELLE mit Werten
Diese Temperaturabhängigkeit der beiden physikalischen Größen wurde in den folgenden beiden Graphen noch einmal veranschaulicht, indem die Werte der Größen gegen die Temperatur aufgetragen wurden. %TODO: Graphen: entsprechender Größenwert gegen Temperatur aufgetragen
Daraus ist zu erkennen, dass die Ladungsträgerdichte ab einer Temperatur von %TODO: WERT & was man sieht.... was steigt / sinkt wann? 



