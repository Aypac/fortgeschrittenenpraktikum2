% !TeX root = ../praktikum.tex
% !TeX encoding = UTF-8
% !Tex spellcheck = de_DE

Wie zu den Messdaten der Gleich- und Wechselstrommessungen wurden die Ladungsträgerdichten und -beweglichkeiten hier für alle verschiedenen aufgenommenen Temperaturwerte von \unit[2]{K} bis \unit[40]{K} bestimmt. In der Abbildung~\ref{fig:temp_mess} ist der Hall-Spannungsverlauf zu verschiedenen Temperaturen abgebildet. Hierbei ist deutlich zu erkennen, wie die SDHO mit steigender Temperatur weniger stark ausgeprägt sind. Die Temperatur, ab welcher der Effekt nicht mehr zu erkennen ist, liegt zwischen $20$ und \unit[40]{K}.

Ladungsträgerdichte und -beweglichkeit wurden wie in den obigen Versuchsteilen über die Hall-Spannung berechnet und die Ergebnisse zu den verschiedenen Temperaturen in Tabelle~\ref{tab:temp_ausw} eingetragen. Die gleichen Ergebnisse sind zusätzlich in Abbildung~\ref{fig:temp_ausw} dargestellt.

\begin{table}[h]
	\centering
	\begin{tabular}{|l|r|l|r|r|}
		\hline
		\multicolumn{1}{|l|}{\cellcolor{black!30} $T$ } & \multicolumn{1}{|l|}{\cellcolor{black!30} $b$ } & \multicolumn{1}{|l|}{\cellcolor{black!30} $U_{xx}$ } & \multicolumn{1}{|l|}{\cellcolor{black!30} $n_s$ } & \multicolumn{1}{|l|}{\cellcolor{black!30} $\mu$ } \\
		\multicolumn{1}{|l|}{\cellcolor{black!30} [$\unit{K}$] } &  \multicolumn{1}{|l|}{\cellcolor{black!30} [\unit{T}] } &
		\multicolumn{1}{|l|}{\cellcolor{black!30} [$\unit{V}$] } &  \multicolumn{1}{|l|}{\cellcolor{black!30} [$\unitfrac{1}{m^2}$] } & \multicolumn{1}{|l|}{\cellcolor{black!30} [$\unitfrac{m^2}{Vs}$] } \\ \hline
		$ 2 $  & $ 8,51422\cdot 10^{-4} $  & $ 2,73230\cdot 10^{-4} $  & $ 7,17948\cdot 10^{15} $  & $ 1,91166\cdot 10^{7} $  \\ 
		$ 5 $  & $ 8,64040\cdot 10^{-4} $  & $ 2,84912\cdot 10^{-4} $  & $ 7,07464\cdot 10^{15} $  & $ 1,86044\cdot 10^{7} $  \\ 
		$ 10 $  & $ 8,47763\cdot 10^{-4} $  & $ 2,80024\cdot 10^{-4} $  & $ 7,21046\cdot 10^{15} $  & $ 1,85726\cdot 10^{7} $  \\ 
		$ 20 $  & $ 8,54649\cdot 10^{-4} $  & $ 2,99456\cdot 10^{-4} $  & $ 7,15237\cdot 10^{15} $  & $ 1,75085\cdot 10^{7} $  \\ 
		$ 40 $  & $ 8,66678\cdot 10^{-4} $  & $ 3,48808\cdot 10^{-4} $  & $ 7,05310\cdot 10^{15} $  & $ 1,52428\cdot 10^{7} $  \\ \hline
	\end{tabular}
	\caption{Berechnete Elektronendichte $n_s$ und -beweglichkeit $\mu$ in Abhängigkeit zu der Probentemperatur, aus Platzgründen ohne Fehler.}
	\label{tab:temp_ausw}
\end{table}


Sowohl die Elektronenbeweglichkeit als auch die Ladungsträgerdichte ist nahezu konstant mit der Temperatur. Beide nehmen tendenziell mit der Temperatur ab. Eine mögliche Erklärung für dieses Verhalten wäre eine erhöhte thermische Streuung. Da der Effekt jedoch sehr klein ist, liegt die Vermutung nahe, dass bei der Messung ein Fehler unterlaufen ist.


\begin{figure}[h]
	\centering
	\includegraphics[scale=1]{graphs/temperatur/auswertung.pdf}
	\caption[Auswertung der Temperaturvariation]{
		Berechnete Elektronendichte $n_s$ und -beweglichkeit $\mu$ in Abhängigkeit zu der Probentemperatur.
	}
	\label{fig:temp_ausw}
\end{figure}

