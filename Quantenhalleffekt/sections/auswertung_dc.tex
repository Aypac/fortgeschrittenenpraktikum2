% !TeX root = ../praktikum.tex
% !TeX encoding = UTF-8
% !Tex spellcheck = de_DE


Anhand der Messdaten aus der Gleichstrommessung, abgebildet in Graphik \ref{fig:full_range_dc} und \ref{fig:2T_range_dc}, wurde die Dichte der Ladungsträger im 2DEG, sowie deren Beweglichkeit bestimmt.

\subsubsection{Näherung über die Hall-Spannung}
\label{ch:naeherung_hall}

Zunächst wurde die Steigung der Hall-Spannung mittels linearer Regression aus dem klassischen Bereich der Messdaten zwischen \unit[-2]{T} und \unit[2]{T} anhand der Formel $ U_{Hall}=a\cdot B $  
bestimmt:
\begin{equation}
	a=\d{U_{Hall}}{B} = (1869,29 \pm 1,10)\cdot10^{-6} \cdot  \unitfrac{m^2}{s}
\end{equation}

Hieraus ließ sich mit der Formel~\eqref{eq:ladungsdichte_steig} und dem bekannten Strom von $I=\unit[100]{nA}$ die Ladungsträgerdichte des 2DEG berechnen: 
\begin{equation}
	n_s= \frac{I}{e}\left({\niced{U_{Hall}}{B}}\right)^{-1}=( 33.435,1 \pm 19,7) \cdot 10^{10}\cdot \unitfrac{1}{cm^2}
\end{equation}
Mit den Bekannten Maßen der Probe wurde anschließend die Beweglichkeit der Ladungsträger bestimmt. Dies erfolgte anhand der Formel \\
\begin{equation}
	\mu=\frac{1}{n_se}\frac{I}{U_{xx}}\frac{L}{W}
\end{equation}
und mit den Probenmaßen von $L=600\mu m$ für die Länge und
 $B=100\mu m$ für die Breite. \\
\begin{equation}
\mu= \unitfrac{cm^2}{Vs} %TODO: ZAHLENWERTE!
\end{equation}
 
\subsubsection{Näherung über die Shubnikov-de Haas-Oszillation}
\label{ch:naeherung_sdho}

Eine Alternative Möglichkeit, die Ladungsträgerdichte zu berechnen, erfolgt über die Shubnikov-de Haas-Oszillation. Hierzu wurde die Längsspannung über der Probe 
%$U_{Längs}$
 gegen $\unitfrac{1}{B}$ aufgetragen und jedem Minimum der Oszillation ein Füllfaktor $\nu$ zugeordnet. Dies ist in Abbildung~\ref{fig:dc_sdho_ausw} zu sehen. 

Mit der Steigung der Geraden von $A=...$  %TODO: ZAHLENWERTE!
ließ sich analog zur Näherung über die Hallspannung mit Gleichung %TODO: REF! 
die Ladungsträgerdichte bestimmen:
\begin{equation}
n_s= \unitfrac{1}{cm^2}   %TODO: ZAHLENWERTE! 
\end{equation}

So ergibt sich aus Gleichung %TODO: REF! 
für die Beweglichkeit:
\begin{equation}
\mu= \unitfrac{cm^2}{Vs}   %TODO: ZAHLENWERTE! 
\end{equation}
 

\begin{figure}[h]
	\centering
	\includegraphics{graphs/dc/auswertung.pdf}
	\caption[Auswertung Füllfaktor Gleichstrommessung]{
		Auswertung Füllfaktor Gleichstrommessung.
	}
	\label{fig:dc_sdho_ausw}
\end{figure}

\begin{equation}
\nu= A\cdot \nicefrac{1}{B}=(13.1462 \pm 0.0190)T \cdot \nicefrac{1}{B}
\end{equation}