% !TeX root = ../praktikum.tex
% !TeX encoding = UTF-8
% !Tex spellcheck = de_DE

In diesem Versuch konnten erfolgreich der Hall- und Shubnikov-de Hass-Effekt nachgewiesen werden. So konnte mit hoher Genauigkeit der Kitzling-Faktor ermittelt werden. Aus dem asymmetrischen Verhalten der Messergebnisse konnte ein Rückschluss auf die Symmetrie der Probe gezogen werden.

Über die Variation verschiedener Parameter konnte weiteres Verständnis gewonnen werden: Das Verhalten des Hall-Streifens unter Rotation konnte sehr anschaulich und in guter Übereinstimmung über eine Geometrische Beziehung verstanden werden. Auch die Variation der Gate-Spannung konnte mit dem Modell eines Kondensators sehr gut und anschaulich verstanden werden. So konnte Rückwirkend plausibel gemacht werden, warum eine Abweichung zwischen DC- und AC-Messung aufgetreten ist. Auch die Temperaturmessung entsprach den Erwartungen.