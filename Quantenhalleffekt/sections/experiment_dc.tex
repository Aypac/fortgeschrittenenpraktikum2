% !TeX root = ../praktikum.tex
% !TeX encoding = UTF-8
% !Tex spellcheck = de_DE

In diesem Versuchsabschnitt wurden die beiden zu untersuchenden Effekte gemessen, während an der Probe ein Gleichstrom anlag. Hierzu wurde gleichzeitig der Spannungsabfall über die Länge und über die Breite der Probe anhand der Software erfasst. An die Probe wurde in x-Richtung ein konstanter Gleichstrom von $\unit[1]{\mu A}$ angelegt. Um eine stabile Temperatur zu erhalten, wurde die Kammer auf \unit[2]{K} geheizt. Jetzt wurde das Magnetfeld auf \unit[-7,7]{T} gefahren. Dies dauert rund sieben Minuten, da es sich um supraleitende Spulen handelt und das Anlegen eines Stroms eine Gegeninduktion verursacht. Sobald das Magnetfeld aufgebaut war, wurde die Hallspannung $U_H$ sowie die Längsspannung $U_{xx}$ aufgezeichnet, während das Magnetfeld mit etwa \unitfrac[1]{T}{min} auf \unit[7,7]{T} gefahren wurde. Mit Hilfe der Formel~\eqref{eq:u2rho} wurden aus diesen Werten die Widerstandswerte berechnet.\\

Die berechneten Widerstände sind in Abbildung~\ref{fig:full_range_dc} zu sehen, die einer analogen Messung im Bereich von $-2$ bis \unit[+2]{T} bei einer konstanten Erhöhung von \unitfrac[0,2]{T}{min} sind in Abbildung~\ref{fig:2T_range_dc} aufgetragen.

\begin{figure}[h]
	\centering
	\includegraphics[scale=1]{graphs/dc/full_range.pdf}
	\caption[Gleichstrommessung im maximalen Magnetfeldbereich]{
		Plot des gemessenen Hall-Widerstand und Shubnikov-de Haas Oszillationen eines mit Gleichstrom durchflossenen 2DES.
	}
	\label{fig:full_range_dc}
\end{figure}


\begin{figure}[h]
	\centering
	\includegraphics{graphs/dc/pm2T_range.pdf}
	\caption[Höher aufgelöste Gleichstrommessung in Magnetfeldteilbereich]{
		Hall-Widerstand und Shubnikov-de Haas Oszillationen eines mit Gleichstrom durchflossenen 2DES.
	}
	\label{fig:2T_range_dc}
\end{figure}


Es sind deutlich die Plateaus des Hall- und die Oszillation des Shubnikov-de Haas-Widerstandes zu erkennen.