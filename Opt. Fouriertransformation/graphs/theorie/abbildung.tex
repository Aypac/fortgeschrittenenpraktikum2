\documentclass[pstricks,border=12pt]{standalone}% remove the border key if you want a tight output without any border.

%Optical elements
%use latex->dvi->ps(->pdf) chain.



\usepackage[utf8]{inputenc}

\usepackage{hyperref}

\usepackage{amsmath} % AMS Math Package
\usepackage{amsfonts}
\usepackage{amsthm} % Theorem Formatting
\usepackage{amssymb}	% Math symbols such as \mathbb
\usepackage{mathtools}
\usepackage{graphicx} % Allows for eps images
\usepackage{multicol} % Allows for multiple columns
\usepackage[]{units}

\usepackage{hyperref} %Links
\usepackage{url}

\usepackage{verbatim} %Fuer mehrzeilige kommentare \begin{comment} \end{comment}

\usepackage[hang]{caption} % Captions einrücken
\usepackage{subfigure}

%Some other shit
\usepackage{float}
%\floatstyle{boxed} %Puts a box around each figure
\restylefloat{figure}
%\usepackage{wrapfig}
\usepackage{microtype}



\usepackage{xparse} % For uses like \DeclareDocumentCommand{\foocmd}{ O{default1} O{default2} m }{#1~#2~#3}

%\usepackage{color}
%\definecolor{gray}{rgb}{.5,.5,.5}
%\definecolor{lightgray}{rgb}{.25,.25,.25}

%Graphics (PGF / TIKZ) **+**+**+**+**+**+**+**+**+**+**+**+**+**+
\usepackage{tikz}
\usetikzlibrary{patterns}
\tikzset{
	hatchhor/.style={pattern=horizontal lines,pattern color=#1},
	hatchhor/.default=black
}
\tikzset{
	hatchvert/.style={pattern=vertical lines,pattern color=#1},
	hatchvert/.default=black
}
\tikzset{
	hatchdiag/.style={pattern=north east lines,pattern color=#1},
	hatchdiag/.default=black
}
\tikzset{
	hatchdiag2/.style={pattern=north west lines,pattern color=#1},
	hatchdiag2/.default=black
}
%also possible grid, crosshatch, dots, crosshatch dots, fivepointed stars, sixpointed stars

\usepackage{pgfplots}
\usepgfplotslibrary{fillbetween}

% Style to select only points from #1 to #2 (inclusive)
\pgfplotsset{select coords between index/.style 2 args={
		x filter/.code={
			\ifnum\coordindex<#1\def\pgfmathresult{}\fi
			\ifnum\coordindex>#2\def\pgfmathresult{}\fi
		}
	}
}


\newcommand{\vasymptote}[2][]{
	\draw [color=gray,densely dashed,#1] ({rel axis cs:0,0} -| {axis cs:#2,0}) -- ({rel axis cs:0,1} -| {axis cs:#2,0});
}

\newcommand{\vertline}[2][]{
	\draw [#1] ({rel axis cs:0,0} -| {axis cs:#2,0}) -- ({rel axis cs:0,1} -| {axis cs:#2,0});
}
%END Graphics **+**+**+**+**+**+**+**+**+**+**+**+**+**+**+**+**+
%Stolen from http://www.dfcd.net/articles/latex/latex.html
% **#**#**#**#**#**#**#**#**#**#**#**#**#**#**#**#**#**#**#**#**#
\makeatletter % Need for anything that contains an @ command

\let\vaccent=\v % rename builtin command \v{} to \vaccent{}
\renewcommand{\v}[1]{\ensuremath{\mathbf{#1}}} % for vectors
\newcommand{\gv}[1]{\ensuremath{\mbox{\boldmath$ #1 $}}} 
% for vectors of Greek letters
\newcommand{\uv}[1]{\ensuremath{\mathbf{\hat{#1}}}} % for unit vector
\newcommand{\abs}[1]{\left| #1 \right|} % for absolute value
\newcommand{\avg}[1]{\left< #1 \right>} % for average
\let\underdot=\d % rename builtin command \d{} to \underdot{}
\renewcommand{\d}[2]{\frac{d #1}{d #2}} % for derivatives
\newcommand{\niced}[2]{\nicefrac{d #1}{d #2}} % for in-text-derivatives
\newcommand{\nicedd}[2]{\nicefrac{d^2 #1}{d #2^2}} % for double derivatives\newcommand{\dd}[2]{\frac{d^2 #1}{d #2^2}} % for in-text-double derivatives
\newcommand{\pd}[2]{\frac{\partial #1}{\partial #2}} 
% for partial derivatives
\newcommand{\pdd}[2]{\frac{\partial^2 #1}{\partial #2^2}} 
% for double partial derivatives
\newcommand{\pdc}[3]{\left( \frac{\partial #1}{\partial #2}
	\right)_{#3}} % for thermodynamic partial derivatives
\newcommand{\ket}[1]{\left| #1 \right>} % for Dirac bras
\newcommand{\bra}[1]{\left< #1 \right|} % for Dirac kets
\newcommand{\braket}[2]{\left< #1 \vphantom{#2} \right|
	\left. #2 \vphantom{#1} \right>} % for Dirac brackets
\newcommand{\matrixel}[3]{\left< #1 \vphantom{#2#3} \right|
	#2 \left| #3 \vphantom{#1#2} \right>} % for Dirac matrix elements
\newcommand{\grad}[1]{\gv{\nabla} #1} % for gradient
\let\divsymb=\div % rename builtin command \div to \divsymb
\renewcommand{\div}[1]{\gv{\nabla} \cdot #1} % for divergence
\newcommand{\curl}[1]{\gv{\nabla} \times #1} % for curl
\let\baraccent=\= % rename builtin command \= to \baraccent
\renewcommand{\=}[1]{\stackrel{#1}{=}} % for putting numbers above =
\newtheorem{prop}{Proposition}
\newtheorem{thm}{Theorem}[section]
\newtheorem{lem}[thm]{Lemma}
\theoremstyle{definition}
\newtheorem{dfn}{Definition}
\theoremstyle{remark}
\newtheorem*{rmk}{Remark}
% **#**#**#**#**#**#**#**#**#**#**#**#**#**#**#**#**#**#**#**#**#


%Equalssign with hat/corresponds to   \equalhat
\newcommand\equalhat{%
	\stackrel{\Lambda}{=}
}
%Equalssign with !
\newcommand\shallbe{%
	\stackrel{!}{=}
}
% := and =:
\newcommand{\defeq}{\vcentcolon=}
\newcommand{\eqdef}{=\vcentcolon}



%Encirecled Numbers, used in Graphics
\let\depth\relax
\def\X#1{%
	%#1%
	%\textcircled{#1}%
	\raisebox{0.9pt}{\textcircled{\raisebox{-.9pt}{#1}}}%
	%\ding{\numexpr171+#1\relax}%
}

% Style to select only points from #1 to #2 (inclusive)
\pgfplotsset{select coords between index/.style 2 args={
		x filter/.code={
			\ifnum\coordindex<#1\def\pgfmathresult{}\fi
			\ifnum\coordindex>#2\def\pgfmathresult{}\fi
		}
	}}
	
	

%\usepackage{tikz}
\usepackage{pst-optexp}


\begin{document}
	
	%\psset{xunit=4cm, yunit=1.6cm}
	\addtopsstyle{Beam}{linecolor=red, beamwidth=0.3, fillstyle=solid, fillcolor=red, opacity=0.25, arrowscale=1.8, linewidth=.01} %,linestyle=none, fillcolor=green!18!white
	
	\begin{pspicture}(1.3,0)(11.6,2) %[showgrid]
	\pnodes(0,1){Begin}(12,1){End}
	
	\psset{mirrortype=extended, mirrordepth=0.15, hatchcolor=black!30}
	
	
	\begin{optexp} %All beams placed behind components
		
		\optplate[plateheight=1.5, abspos=0](Begin)(End){~}
		\pinhole[abspos=2, phwidth=0.05, innerheight=0.08,labeloffset=.95, labelangle=200](Begin)(End){Objekt}
		\lens[lensradius=2,lensheight=1.2, labeloffset=0.9, labelangle=180, abspos=4,n=1.5](Begin)(End){Linse}
		\optplate[plateheight=1.2, abspos=6,labeloffset=.9, labelangle=180](Begin)(End){Fourierebene}
		\lens[lensradius=2,lensheight=1.2, labeloffset=0.9, labelangle=180, abspos=8,n=1.5](Begin)(End){Linse}
		\optplate[plateheight=1.2, abspos=10,labeloffset=.95, labelangle=160](Begin)(End){Abbildungsebene}
		
		\drawwidebeam[beamwidth=0.6, beamdiv=0]{1-2}
		
		\drawwidebeam[beamwidth=0.08, beamdiv=0, beamangle=10, opacity=0.1, linecolor=red!70]{2-}
		\drawwidebeam[beamwidth=0.08, beamdiv=0, beamangle=-10, opacity=0.1, linecolor=red!70]{2-}
		\drawwidebeam[beamwidth=0.1, beamdiv=0]{2-}
	
	
	\end{optexp}
	
	\psset{arrows=|*-|*}
	\pcline([offset=-1]\oenodeCenter{2})([offset=-1]\oenodeCenter{3})
	\ncput*{$f$}
	\pcline([offset=-1]\oenodeCenter{3})([offset=-1]\oenodeCenter{4})
	\ncput*{$f$}
	\pcline([offset=-1]\oenodeCenter{4})([offset=-1]\oenodeCenter{5})
	\ncput*{$f$}
	\pcline([offset=-1]\oenodeCenter{5})([offset=-1]\oenodeCenter{6})
	\ncput*{$f$}
	
	\end{pspicture}
	
	
	
	\begin{comment}
		\optbox[position=start,compname=LA](L0)(L2){Laser}
		\optbox[position=0.53, optboxsize=0.2 0.6,labelangle=190, labeloffset=0.65](L0)(L2){{\small ND-Filter}}
		\optbox[position=0.47, optboxsize=0.2 0.6](L0)(L2){~}
		\mirror[compname=M1](L0)(L2)(L3)
		\mirror[compname=M2](L2)(L3)(L5)
		\optretplate[beam](L3)(L5){$\nicefrac{\lambda}{2}$}
		\beamsplitter[compname=BS](L3)(L5)(L6)
		
		\mirror[compname=M3,variable](L5)(L6)(L7)
		\mirror[compname=M4,variable](L7)(L8)(L9)
		\optretplate[beam](L8)(L9){$\nicefrac{\lambda}{2}$}
		
		\drawbeam[linecolor=red,arrows=->] {BS}(-4,3.5)
		
		\drawbeam[linecolor=red] {LA}{M1}{M2}{BS}{M3}{M4}{EK}
		
	
		%\optsource[innerlabel,beamdiv=1,beamangle=0,compname=LA](L)(C){Laser}
		\lens[lensradius=1.5,lensheight=1.4, labeloffset=1, labelangle=180](L1)(L2){Lens 1}
		\pinhole[innerheight=0.025,phwidth=-0.1, outerheight=1.4, labeloffset=1](L2)(L3){Pinhole/CCD}
		\lens[lensradius=3,lensheight=1.4, labeloffset=1](L3)(L4){Lens 2}
		
		\optplate[plateheight=1.5, labelangle=-140, labeloffset=1.3, compname=CCD](L4)(L5){Screen/CCD}
		%\optbox[position=end, labeloffset=0,labelref=relative,compname=CCD](L4)(L5){CCD}
		
		\drawwidebeam[beamdiv=5.5, fillcolor=orange, linecolor=white!]{1-5}
		\drawwidebeam[beamdiv=1.5, ArrowInside=->]{-}
	\end{comment}
	
\end{document}