% !TeX root = ../praktikum.tex
% !TeX encoding = UTF-8
% !Tex spellcheck = de_DE



Durch diesen Versuch werden die theoretischen Grundlagen der Fouriertransformation anhand anschaulicher Experimente nachvollziehbar. 
Nützlich für das Verständnis ist dabei ein Vergleich der Eigenschaften von Linse
und Aufbau mit den erforderlichen Eigenschaften einer Funktion (Linearität, Ähnlichkeit, Verschiebung, Faltung) für eine Fouriertransformation.
Insbesondere wird dies durch die Verwendung von verschiedenen Filtern und die daraus folgende Manipulation der Bilder vermittelt. 

Der selbständige Versuchsaufbau, der die Optimierung des Diodenlasers als auch des Strahlengangs umfasste, veranschaulichte deutlich die Herausforderungen des Versuchs, die Herausforderungen beim Experimentieren und dem Umgang mit der Ausrüstung in der Optomechanik. Dabei ist viel Geduld, z.B. bei der Einkopplung erforderlich, sowie auch Kreativität, was uns anhand des Baus der Photodiode zur Messung der Effizienz des Lasers verdeutlicht wurde. Weiterhin zeigte das Justieren des Strahlengangs, dass bereits kleinste Veränderungen einen enormen Einfluss auf die Auflösung der Abbildung haben können.
%Die entstehenden Unterschiede können anhand unserer Abbildungen des Fourierhauses gezeigt werden, die an zwei unterschiedlichen Tagen erzeugt wurden (Abb.19 vgl. Abb. 20%TODO: REF!!!).
Die Wirkung der Filter auf die Abbildungen erhöhte ebenfalls das Verständnis für Effekte, wie Weichzeichner oder Kantenerkennung in der digitalen Bildverarbeitung. Da Filterungen ebenfalls in der Akustik verwendet werden, um z.B. Rauschen herauszufiltern, oder zur Kompression von Daten (jpeg, mp3), wurde somit die breite Anwendung der Fouriertransformation anhand des Versuchs der optischen Fouriertransformation verdeutlicht.