% !TeX root = ../praktikum.tex
% !TeX encoding = UTF-8
% !Tex spellcheck = de_DE

Die geometrische Optik ist der Teilbereich der Optik, wo Lichtwellen durch idealisierte Strahlen angenähert werden um den Weg des Lichtes zu (re)konstruieren. Sämtliche Schlussfolgerungen basieren auf diesen vier Axiomen:

\begin{enumerate}
	\item{Axiom:} In homogenem Material verlaufen Lichtstrahlen gerade.
	\item{Axiom:} An der Grenze zwischen zwei homogenen und isotropen Materialien wird das Licht nach dem Reflexionsgesetz reflektiert und nach dem Brechungsgesetz gebrochen.
	\item{Axiom:} Zeit- bzw. Strahlenumkehr, die Richtung eines Lichtstrahles ist belanglos.
	\item{Axiom:} Die Lichtstrahlen beeinflussen sich nicht gegenseitig.
\end{enumerate}

Für Linsen ergibt sich eine Brennweite 