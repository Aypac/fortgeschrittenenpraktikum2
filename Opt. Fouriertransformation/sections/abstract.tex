% !TeX root = ../praktikum.tex
% !TeX encoding = UTF-8
% !Tex spellcheck = de_DE

Die Fourier-Analytik und insbesondere die Fouriertransformation sind auf vielen modernen Anwendungen kaum noch weg zu denken. Sie ist essentieller Bestandteil vieler Bildverarbeitungsalgorithmen\cite{easton_fourier_2010} und -kompressionsverfahren wie JPEG\cite{_jpeg_2015}, sie wird genutzt um Bildinformationen Computeralgorithmen zugänglich zu machen\cite{prof._dr._norbert_link_vorlesungsscript:_????} und in vielen anderen Bereichen der Signalverarbeitung.\\

In diesem Versuch soll mit einfachen optischen Mitteln eine Fouriertransformation auf Bildern durchgeführt, die Fouriertransformierte manipuliert und rücktransformiert werden. Dabei wird dem Leser ein intuitives Verständnis für die Funktionsweise und Bedeutung von Fouriertransformationen vermittelt werden.