% !TeX root = ../praktikum.tex
% !TeX encoding = UTF-8
% !Tex spellcheck = de_DE

Die Fourier-Analytik und insbesondere die Fouriertransformation sind auf vielen modernen Anwendungen kaum noch weg zu denken. Sie ist essentieller Bestandteil vieler Bildverarbeitungsalgorithmen \cite{easton_fourier_2010} und -kompressionsverfahren \cite{_jpeg_2015}, sie wird genutzt um Bildinformationen Computeralgorithmen zugänglich zu machen \cite{prof._dr._norbert_link_vorlesungsscript:_????}. Auch in vielen anderen Bereichen der Signalverarbeitung hat sie eine große Bedeutung. Weitere Anwendungen werden durch die Kombination mit dem Schlierenverfahren, z.B. in der Luftfahrt oder bei der Motorenentwicklung, da mit diesem Gas- und Flüssigkeitsströmungen sichtbar gemacht werden können.\\

In diesem Versuch wird mit einfachen optischen Mitteln eine Fouriertransformation an Bildern durchgeführt und die Fouriertransformierte manipuliert. Dabei wird ein intuitives Verständnis für die Funktionsweise und Bedeutung von Fouriertransformationen vermittelt.