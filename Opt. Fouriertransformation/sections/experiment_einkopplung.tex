% !TeX root = ../praktikum.tex
% !TeX encoding = UTF-8
% !Tex spellcheck = de_DE

Für die Durchführung des Versuchs wurde ein temperaturgesteuerter cw- Laser mit emittiertem Licht der Wellenlänge 660nm verwendet. Der emittierte Laserstrahl wird in eine Leitungsfaser eingekoppelt.
\\
\begin{figure}[h]
\centering
\includegraphics[width=0.5\linewidth]{graphs/versuchsaufbau/lasereinheit}
\caption{Schematischer Strahlungsaufbau zwischen Laser und Fasereinkopplung. Bei dem Laser handelt es sich um das Modell  \textit{LD: Mitsubishi ML101J27}. Betrieben wurde der Laser mit 90,3 mA bei $18^\circ C$ und hat eine maximale Leistung von 35 mW. Mit Hilfe der $\frac{\lambda}{2}$- Platte und des Strahlteilers dahinter kann die Intensität des Laserstrahls eingestellt werden. Der zweite Strahl nach dem Strahlteiler dient zum parallelen Durchführen des Versuchs durch eine zweite Gruppe und wird daher nicht weiter betrachtet.}
\label{fig:lasereinheit}
\end{figure}

Der Faserkopplungsaufbau (siehe Abbildung \ref{fig:lasereinheit}) wird nicht selbst aufgebaut, sondern lediglich optimiert. Zur Regulierung der Intensität wird die Eigenschaft der Polarisation des Lichts ausgenutzt, welche anschaulich als \textit{Schwingungsebene} einer Lichtwelle beschrieben werden kann. Der Laserstrahl durchläuft eine $\frac{\lambda}{2}$- Platte; dabei handelt es sich um eine doppelbrechende Platte, die den beiden entstehenden Teilwellen einen Gangunterschied erteilt, der gleich der Hälfte der Bezugswellenlänge $\lambda$ ist. In Diagonalstellung wird die Polarisation des Lichts gedreht. Letzterer Effekt wird hier genutzt, da so in Kombination mit dem Strahlteiler hinter der  $\frac{\lambda}{2}$- Platte die Intensität des anschließend verwendeten Lichts reguliert werden kann. Durch Drehung der  $\frac{\lambda}{2}$- Platte vor dem Strahlteiler kann beeinflusst werden, wie hoch der Anteil des Lichts mit der Polarisation ist, welche durch den Strahlteiler zum restlichen Versuchsaufbau gelenkt wird.   \\
Auch vor der Fasereinkopplung spielt die Polarisation eine wichtige Rolle, da die lichtleitende Faser für eine bestimmte Polarisation die höchste Effizienz aufweist. \\

Zur Optimierung der Einkopplung des Lichts in die Faser wird ein Laserpointer an dem noch freien Ende der Faser angebracht und vor dem Einkoppler mithilfe der Spiegel eine optimale Überlagerung der beiden Signale eingestellt. \\
Anschließend betrachtet man das eingekoppelte Signal am Ende der Faser, um dessen Leistung weiter zu optimieren. Dies erfolgt zunächst mit dem bloßen Auge und anschließend mit einem Powermeter, welches an ein Oszilloskop angeschlossen wird, um schnelle Änderungen der gemessenen Lichtleistung besser sichtbar zu machen. Es wird ein Optimum der Fasereinkopplung möglichst genau eingestellt. 



