% !TeX root = ../praktikum.tex
% !TeX encoding = UTF-8
% !Tex spellcheck = de_DE


Hinter dem Faserauskoppler wird der Aufbau aus Abbildung ... realisiert.
 \begin{comment} Abbildung und Nummerierung noch ergänzen!!! \end{comment} 
Dabei wurden Linsen mit den Brennweiten $f_{1}=20mm$ , $f_{2}=300mm$ , $f_{3}=100mm$ und $f_{4}=100mm$ verwendet. \\
Der Spiegel hinter dem Auskoppler wurde zusätzlich in den 4f-Aufbau aufgenommen, um den Verlauf des Laserstrahls im optischen Pfad besser feinjustieren zu können. Ebenso wurde das Pinhole ergänzt, um das Gaußprofil des zur Messung verwendeten Laserstrahls zu optimieren. \\

In dem 4f-Aufbau passiert der Laserstrahl nach der Reflektion am ersten Spiegel eine $\frac{\lambda}{2}$- Platte und dahinter einen Strahlteiler. Mithilfe dieser beiden Komponenten kann die Intensität des anschließend verwendeten Strahls eingestellt werden. \\
Um die abzubildenden Objekte vollständig ausleuchten zu können, wird der Laserstrahl mithilfe der ersten beiden Linsen aufgeweitet und kollimiert. Im Brennpunkt der dritten Linse befindet sich ein Objektträger in der Gegenstandsebene. In diesem werden die abzubildenden Objekte angebracht. \\
Die Fourierebene befindet sich im Brennpunkt der Linsen 3 und 4. Nach der vierten Linse wird der Strahl erneut kollimiert und trifft auf die CCD Kamera, Kamera 1. \\
Um Aufnahmen der Fourierspektren zu erhalten, wird bei Bedarf eine zweite Kamera, Kamera 2, in der Fourierebene montiert. So wird gleichzeitig der 2f-Aufbau realisiert. \\ 

Nachdem der 4f-Aufbau montiert und der Verlauf des Laserstrahls im optischen Pfad optimiert ist, werden nacheinander die Objekte 1 bis 5 (Siehe Abbildung ...) in Form von Dias in dem Objektträger montiert. Mit den beiden Kameras werden nacheinander für jedes der Objekte Aufnahmen in der Abbildungsebene und zugehörig zu jeder Einstellung auch in der Fourierebene gemacht. Zudem werden für die Objekte 4 und 5 beispielhaft verschiedene Filter in die Fourierebene gestellt und Aufnahmen der Kamera in der Abbildungsebene gemacht. Für Objekt 4 wird hierzu ein Tiefpass und mehrere Breitbandfilter verwendet. Für Objekt 5 wird ein Halbebenenfilter horizontal, vertikal und diagonal in die Fourierebene gehalten. \\

Als Nächstes wird die Abbildung eines Fingerabdrucks auf einem Glasplättchen zunächst ohne Filter aufgenommen. Anschließend wird die Abbildung mit einem in der Fourierebene befindlichen Hochpass- und einem Halbebenenfilter aufgenommen. Zudem wird mit Kamera 2 das Fourierspektrum des Fingerabdrucks photographiert. \\

Als Letztes wird ein Teelicht auf die Position des Objektträgers gestellt und ein Halbebenenfilter in der Fourierebene installiert. Mit Kamera 1 werden mehrere Abbildungen aufgenommen, um die Strömungsbewegungen oberhalb der Flamme beobachten zu können. 
Zum Vergleich wird zudem eine Aufnahme mit Halbebenenfilter, jedoch ohne Teelicht gemacht. 





