% !TeX root = ../praktikum.tex
% !TeX encoding = UTF-8
% !Tex spellcheck = de_DE

Das mit Hilfe der Linse transformierte Bild in der Fourierebene enthält räumlich die Bildfrequenzinformationen. Wenn nun Teile dieses Bildes hervorgehoben oder ausgelöscht werden, können ebenfalls die Frequenzinformationen des rücktransformierten Bildes manipuliert werden. Dies wird insbesondere in der Bildbearbeitung genutzt. So können beispielsweise periodische Streifen aus Bildübertragungsfehlern einer Raumsonde entfernt werden, Abbildung~\ref{fig:Plato_LO_big}.

\begin{figure}[h]
	\centering
	\includegraphicsRS[width=0.8\textwidth][.8]{images/Plato_LO_big.jpg}
	\caption[Beispiel der FT in der Bildbearbeitung]{Bild der \textit{Lunar Orbiter} Mission (links) und ein mittels einer Manipulation des Fourierbildes erzeugtes verbessertes Bild (rechts) \cite{_highlight_????}.}
	\label{fig:Plato_LO_big}
\end{figure}

Man unterscheidet verschiedene Filtertypen: Ein Hochpassfilter entfernt Frequenzen unterhalb einer Grenze und lässt darüber liegende hindurch. Ein Tiefpassfilter entfernt Frequenzen oberhalb einer gewissen Grenze. Ein Bandfilter ist im wesentlichen eine Kombination aus Hoch- und Tiefpassfilter und lässt nur Frequenzen eines bestimmten Frequenzbereiches hindurch. Jedoch können auch andere Räumliche Filter sinnvoll sein: Für das Bild in Abbildung~\ref{fig:Plato_LO_big} beispielsweise wurden gezielt die Bereiche entfernt, der die horizontale periodische Information (die Streifen) enthalten. Diese Filter können dank der Darstellung der Frequenzinformation in der Fourierebene im Versuch räumlich realisiert werden.